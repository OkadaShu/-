\documentclass[a4j,11pt]{jsarticle}
\usepackage{semi}
\usepackage{breqn}
\usepackage{tabularx}
\usepackage{listings,jlisting}

\lstset{
  basicstyle={\ttfamily},
  identifierstyle={\small},
  commentstyle={\smallitshape},
  keywordstyle={\small\bfseries},
  ndkeywordstyle={\small},
  stringstyle={\small\ttfamily},
  frame={tb},
  breaklines=true,
  columns=[l]{fullflexible},
  numbers=left,
  xrightmargin=0zw,
  xleftmargin=3zw,
  numberstyle={\scriptsize},
  stepnumber=1,
  numbersep=1zw,
  lineskip=-0.5ex
}

\newcolumntype{Y}{&gt;{\centering\arraybackslash}X} %中央揃え
 
\makeatletter % プリアンブルで定義開始

% 図番号を"<章節などの番号番号> - <図番号>" へ
\renewcommand{\thefigure}{\thesubsection-\arabic{figure}}
\renewcommand{\thetable}{\thesubsection-\arabic{table}}
% 章が進むごとに図番号をリセットする
\@addtoreset{figure}{subsection}
\@addtoreset{table}{section}
\renewcommand{\theequation}{% 式番号の付け方
	\thesection.\arabic{equation}}
	\@addtoreset{equation}{section}
\makeatother % プリアンブルで定義終了

\renewcommand{\headfont}{\bfseries} %章タイトルなどを明朝体にする

\makeindex
\begin{document}
\definecolor{cellcolor}{rgb}{ 1, .90, .90}
\definecolor{rowcolor}{rgb}{.85, .85, 1}
\setcounter{tocdepth}{3}
\thispagestyle{empty}
\clearpage
\addtocounter{page}{-1}
\begin{center}

\huge
2020年度 卒業論文\\[60pt]
\HUGE
逆畳込みとケプストラム法を\\
組み合わせることによる\\
残響除去の提案\\[65pt]
\huge
指導教員 須田 宇宙 准教授\\[40pt]
千葉工業大学 情報ネットワーク学科\\[10pt]
須田研究室\\[40pt]
1632038 \hspace{70pt} 岡田 秀\\[110pt]
\end{center}
\begin{flushright} 
\huge

\textcolor{white}{文字}

\textcolor{white}{文字}

提出日 2020年1月24日
\end{flushright}
\newpage
\thispagestyle{empty}
\clearpage
\addtocounter{page}{-1}
\large
% 目次
\tableofcontents
\thispagestyle{empty}
\clearpage
\addtocounter{page}{-1}
\newpage
%表目次
\listoftables
%図目次
\listoffigures
\thispagestyle{empty}
\clearpage
\addtocounter{page}{-1}
\begin{comment}

\end{comment}
\newpage
\end{document}
