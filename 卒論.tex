\documentclass[a4j,12pt]{jarticle}
\usepackage[dvipdfmx]{graphicx}
\usepackage{amssymb}
\usepackage{amsmath}
\usepackage{float}
\begin{document}
\begin{center}
\thispagestyle{empty}
\vspace*{5zh}
\huge
令和2年度 卒業論文\\[50pt]
{\Huge 屋内におけるインパルス応答を用いた反射音除去の一提案}\\
[80pt]
\huge
指導教員 須田 宇宙 准教授\\[30pt]
千葉工業大学 情報ネットワーク学科\\[10pt]
須田研究室\\[60pt]
1632038 \hspace{70pt} 岡田 秀\\[75pt]
\end{center}
\vspace*{-2cm}
\begin{flushright} 
\huge
提出日 2020年1月--日
\end{flushright}

\newpage
\pagenumbering{roman}
\tableofcontents
\newpage
\pagenumbering{arabic}
\section{緒言}
目次を作る際は\verb+\tableofcontents+ と打ちます。\\
新しいページに区切るときは\verb+\newpage+ と打ちます

%背景+問題点
近年,災害が多発している中,町内放送などの非常放送が重要性を増している.屋外,デパートなどの屋内,トンネルの中など様々な場所で館内放送や非常放送の整備されている.しかし,スピーカから離れている地点で放送を聞くと建物などによる反射音により,内容を聞き取りづらいことが多い.ここで,信号処理によって反射音を除去できれば,放送内容を聞き取ることが可能となる.

火災が発生した場合,屋内のスピーカからの非常放送を屋内で受聴することになるが,反射や残響によって聞き取りづらいことが懸念される.
一方,建築音響の分野では,単発の信号音に対する反射音の時間遅れとその大きさを表す指標としてインパルス応答が用いられる.反射音を除去する方法として佐藤\cite{oka1}の研究では屋外での非常用放送の音声の先頭に信号音を付加し,その信号音を基準としてリアルタイムにインパルス応答を計算し,余分な反射音を除去する手法を提案した.しかし,理論は正しかったが,実際の音声を用いて計算したところうまく動作しなかった.

%目的
そこで本研究では,佐藤\cite{oka1}の研究を先行研究として取り入れ,実際の音声でも動作し,非常放送を明瞭に聞き取ることができるようにすることを目的とする.

\newpage

\section{先行研究}

非常放送には反射音が建物などによる付加してしまい内容を聞き取りづらいといったことが問題点として挙げられており,その反射音を除去するために佐藤はクロススペクトル方を用いて反射音の除去に取り組んだ.

手法としては元波形の前におらかじめ信号音を付加させておく.それをもとに反射度合いを推定するため,インパルス応答を求める.反射音の付加された音声をインパルス応答で割ることで反射音が除去できる.

計算内容としては元波形をS,反射音の含まれた音声をRとすると,
はじめに両波形をフーリエ変換することにより周波数成分を求める.

\begin{equation}
  S(f) = \mathcal{F}[S(t)]
\end{equation}

\begin{equation}
  R(f) = \mathcal{F}[R(t)]
\end{equation}
 
元波形の周波数成分を二乗する(3)ことでパワースペクトルP(f),元波形の周波数成分と反射音の含んだ波形の周波数成分の積(4)によりクロススペクトルC(f)を求めることができる.
\begin{equation}
  P(f) = [S(f)]^2
\end{equation}

\begin{equation}
  C(f) = [S(f)]*[R(f)]
\end{equation}

クロススペクトルをパワースペクトルで除算すること(5)により伝達関数H(f)を求めることができ,伝達関数を逆フーリエ変換すること(6)によりインパルス応答IR(t)を求めることができる.
\begin{equation}
  H(f) = \frac{C(f)}{P(f)}
\end{equation}

\begin{equation}
  IR(t) = \mathcal{F}^{-1}[H(f)]
\end{equation}

求めたインパルス応答を用いて,反射音の付加している音声の周波数成分をインパルス応答で除算すること(6)により反射音の除去された音声の周波数成分N(f)を求めることができる.またそれを逆フーリエ変換すること(7)により反射音の除去された音声N(t)を取り出すことができる.
\begin{equation}
  N(f) = \frac{R(f)}{IR(t)}
\end{equation}

\begin{equation}
  N(t) = \mathcal{F}^{-1}[N(f)]
\end{equation}

\begin{equation}
  N(t)\fallingdotseq R(t)
\end{equation}


\section{行政放送}
\subsection{自然災害による放送}
\subsection{行政放送などの公共放送における問題点}
Lorem ipsum dolor sit amet, consectetur adipisicing elit, sed do eiusmod tempor incididunt ut labore et dolore magna aliqua. Ut enim ad minim veniam, quis nostrud exercitation ullamco laboris nisi ut aliquip ex ea commodo consequat. Duis aute irure dolor in reprehenderit in voluptate velit esse cillum dolore eu fugiat nulla pariatur. Excepteur sint occaecat cupidatat non proident, sunt in culpa qui officia deserunt mollit anim id est laborum.

\section{信号処理}
\subsection{周波数特性.周波数領域とは}
\subsection{離散フーリエ変換}
\subsection{反射音・残響音}
\subsection{伝達関数.インパルス応答}
\subsection{畳み込み・逆畳込み}
\subsection{ケプストラム法}
Lorem ipsum dolor sit amet, consectetur adipisicing elit, sed do eiusmod tempor incididunt ut labore et dolore magna aliqua. Ut enim ad minim veniam, quis nostrud exercitation ullamco laboris nisi ut aliquip ex ea commodo consequat. Duis aute irure dolor in reprehenderit in voluptate velit esse cillum dolore eu fugiat nulla pariatur. Excepteur sint occaecat cupidatat non proident, sunt in culpa qui officia deserunt mollit anim id est laborum.
\begin{equation}
Y=ax^3+bx^2+cx+d \\
\end{equation}
\begin{eqnarray}
E[3^S] & = & \sum ^{N} _{k=0}{} _{N}\mathrm{C}_{k} \left( \frac{3}{9} \right)^{k} \left( \frac{8}{9} \right)^{N-k} \\
       & = & \left( \frac{11}{9}\right)^N \\
\end{eqnarray}
\subsubsection{サブサブセクション}
Lorem ipsum dolor sit amet, consectetur adipisicing elit, sed do eiusmod tempor incididunt ut labore et dolore magna aliqua. Ut enim ad minim veniam, quis nostrud exercitation ullamco laboris nisi ut aliquip ex ea commodo consequat. Duis aute irure dolor in reprehenderit in voluptate velit esse cillum dolore eu fugiat nulla pariatur. Excepteur sint occaecat cupidatat non proident, sunt in culpa qui officia deserunt mollit anim id est laborum.
\begin{eqnarray}
\int^\frac{\pi}{2} _0 \sin^4\theta \cos^2 \theta {\rm d} \theta &=& \frac{1}{2} \cdot 2 \int^\frac{\pi}{2} _0 \sin^{2 \cdot \frac{5}{2} -1}\theta \cos^{2 \cdot \frac{3}{2}-1} \theta {\rm d}\theta\\
&=& \frac{1}{2} \cdot \rm B \left( \frac{5}{2} , \frac{3}{2} \right)\nonumber \\
&=& \frac{\Gamma \left(\frac{5}{2} \right)\Gamma \left(\frac{3}{2} \right)}{2\cdot\Gamma \left( 4 \right)}\\
&=& \frac{\frac{3}{2} \cdot \frac{1}{2} \cdot \sqrt{\pi} \cdot \frac{1}{2} \cdot \sqrt{\pi}}{2 \cdot 3! }  =  \frac{3 \cdot \pi}{2^5 \cdot 3}\nonumber \\
&=& \frac{\pi}{32}
\end{eqnarray}
\section{反射音除去の一提案}
\subsection{先行研究による手法}
\subsection{新たな手法提案}
\section{実験と検証}
\subsection{調査目的と調査方法}
\subsection{回答の集計と効果の検証}
Lorem ipsum dolor sit amet, consectetur adipisicing elit, sed do eiusmod tempor incididunt ut labore et dolore magna aliqua. Ut enim ad minim veniam, quis nostrud exercitation ullamco laboris nisi ut aliquip ex ea commodo consequat. Duis aute irure dolor in reprehenderit in voluptate velit esse cillum dolore eu fugiat nulla pariatur. Excepteur sint occaecat cupidatat non proident, sunt in culpa qui officia deserunt mollit anim id est laborum.
\subsection{考察}
Lorem ipsum dolor sit amet, consectetur adipisicing elit, sed do eiusmod tempor incididunt ut labore et dolore magna aliqua. Ut enim ad minim veniam, quis nostrud exercitation ullamco laboris nisi ut aliquip ex ea commodo consequat. Duis aute irure dolor in reprehenderit in voluptate velit esse cillum dolore eu fugiat nulla pariatur. Excepteur sint occaecat cupidatat non proident, sunt in culpa qui officia deserunt mollit anim id est laborum.

\section{結言}
Lorem ipsum dolor sit amet, consectetur adipisicing elit, sed do eiusmod tempor incididunt ut labore et dolore magna aliqua. Ut enim ad minim veniam, quis nostrud exercitation ullamco laboris nisi ut aliquip ex ea commodo consequat. Duis aute irure dolor in reprehenderit in voluptate velit esse cillum dolore eu fugiat nulla pariatur. Excepteur sint occaecat cupidatat non proident, sunt in culpa qui officia deserunt mollit anim id est laborum.

\section*{謝辞}
\addcontentsline{toc}{section}{謝辞}
ここに研究の謝辞.主にご協力いただいた方など.
\bibliographystyle{jplain}
\addcontentsline{toc}{section}{参考文献}
\begin{thebibliography}{99}

\bibitem{oka1}佐藤 由希子,"インパルス応答を用いた反射音除去のための一提案", 千葉工業大学卒業論文, 2014
\bibitem{oka2}小泉 宣夫,"基礎音響・オーディオ学",コロナ社 2005 36-38頁
\bibitem{1}
谷啓(2011),
統計学的トロンボーン演奏法.
どこかの統計学論文誌A, 0号, Vol.0, 11--92 
\bibitem{2}
つのだ☆ひろ(2009),
Rを使ったドラム演奏法.
どこかの統計学論文誌B, 0号, Vol.0, 11--92 
\bibitem{3}
Dan Aykroyd(2000),
Statistical American Joke.
{\it Journal of Blues Brothers},0号, Vol.0, 11--92 

\bibitem{4}
「ホームページの引用はあんまりしないほうがいいよ講座」{\it http://www.google.co.jp/}(最終アクセス 2013年1月2日)
\end{thebibliography}

\section*{付録}
\addcontentsline{toc}{section}{付録}

\end{document}
